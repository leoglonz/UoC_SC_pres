\documentclass{beamer}
\hypersetup{pdfpagemode=FullScreen}

\usetheme{cambridge} 

%%%%%%%%%%%%%%%%%%%%%%%%%%%%%%%%%%%%%%%

\usepackage[natbib=true,style=authoryear,backend=bibtex,useprefix=true]{biblatex}
\setbeamercolor*{bibliography entry title}{fg=black}
\setbeamercolor*{bibliography entry location}{fg=black}
\setbeamercolor*{bibliography entry note}{fg=black}
\setbeamertemplate{bibliography item}{}
\renewcommand*{\bibfont}{\scriptsize}
\addbibresource{uoc-qo.bib}

% adding navigation tools:
\setbeamertemplate{navigation symbols}{} 

\addtobeamertemplate{footline}{
    \leavevmode%
    \hbox{%
    \begin{beamercolorbox}[wd=\paperwidth,ht=2.75ex,dp=.5ex,right,rightskip=1em]{mycolor}%
\usebeamercolor[fg]{navigation symbols}\insertslidenavigationsymbol%
\insertframenavigationsymbol%
\insertsubsectionnavigationsymbol%
\insertsectionnavigationsymbol%
\insertdocnavigationsymbol%
\insertbackfindforwardnavigationsymbol%
    \end{beamercolorbox}%
    }%
    \vskip0.5pt%
}{}


\definecolor{myblue}{RGB}{33,84,157}
\setbeamercolor{footline}{fg=myblue}
\setbeamerfont{footline}{series=\bfseries}
%%%%%%%%%%%%%%%%%%%%%%%%%%%%%%%%%%%%%%%

  
%%%%%%%%%%%%%%%%%%%%%%%%%%%%%%%%%%%%%%%
%%% Standard packages:
\usepackage[english]{babel}
\usepackage[latin1]{inputenc}
\usepackage{graphicx}
\usepackage{multicol}
\usepackage{subfigure}
\usepackage{listings}
\usepackage[UKenglish]{isodate}
%%%%%%%%%%%%%%%%%%%%%%%%%%%%%%%%%%%%%%%


%%%%%%%%%%%%%%%%%%%%%%%%%%%%%%%%%%%%%%%
% Setup TikZ
\usepackage{tikz}
\usetikzlibrary{arrows}
\tikzstyle{block}=[draw opacity=0.7,line width=1.4cm]

\cleanlookdateon
%%%%%%%%%%%%%%%%%%%%%%%%%%%%%%%%%%%%%%%


%%%%%%%%%%%%%%%%%%%%%%%%%%%%%%%%%%%%%%%
% Author, Title, etc.
\title[Quantum Oscillations] 

}
% {%
%  {\bf Quantum Oscillations in\\ Unconventional Superconductors}%}
% }

\author[Lonzarich]
{
  \underline{Leo~Lonzarich} \and Gilbert~Lonzarich\inst{1} \and Suchitra~Sebastian\inst{1} \and Piers~Coleman\inst{2}
  }

\institute[Cambridge]
{
% \inst{1}%
% Grinnell College, USA\\
\inst{1}%
Cavendish Laboratory, University of Cambridge, UK\\
\inst{2}%
Materials Theory Group, Rutgers University, USA
}

\date[CMS2021]
{Condensed Matter Group Summer 2021 \hfill \today}
%%%%%%%%%%%%%%%%%%%%%%%%%%%%%%%%%%%%%%%


\begin{document}

\begin{frame}
  \titlepage
\end{frame}

\begin{frame}{Outline}
  \tableofcontents
\end{frame}


% #########################################
% #########################################

\section{Defining the Problem}
\subsection{SC Classifications}

\begin{frame}
{Superconductor Classifications}
    The well-established...
    \begin{itemize}
        \item \textsc{Gen I}:
            \begin{itemize}
                \item Charge neutral, e.g. lead
                \item Traditional Cooper pair
            \end{itemize}
        \item \textsc{Gen II}
            \begin{itemize}
                \item Magnetic moment / local pairing of electrons
                \item Chemical doping, disabling long-range ferromagnetic effects
            \end{itemize}
    \end{itemize}
    The frontier...
    \begin{itemize}
        \item \textsc{Gen III}: % Hidden SCs!!!!!!!!!!!!!!!
            \begin{itemize}
                \item Sodium chloride with charge separation, ionic SCs
                \item Borderline ionic insulators
            \end{itemize}
        \item \textsc{Gen IV}:
            \begin{itemize}
                \item Multiferroics / magnetism and ionics
            \end{itemize}
    \end{itemize}
        
        
    % But where is the 'unconventional'?
 
\end{frame}


% #########################################
% #########################################


\subsection{States in the Gap}

\begin{frame}
{States in the Gap}

        

\end{frame}


% #########################################
% #########################################


\subsubsection{Quantum Oscillations}

\begin{frame}[t]{Quantum Oscillations}
  \begin{block}{Inputs}
    \begin{itemize}
    \item A \alert{genotype matrix} $G$.
    \item The \alert{rows} of the matrix are \alert{taxa / individuals}.
    \item The \alert{columns} of the matrix are \alert{SNP sites /
        characters}. 
    \end{itemize}
  \end{block}
  \begin{block}{Outputs}
    \begin{itemize}
    \item A \alert{haplotype matrix} $H$.
    \item Pairs of rows in $H$ \alert{explain} the rows of $G$.
    \item The haplotypes in $H$ are \alert{biologically plausible}. 
    \end{itemize}
  \end{block}
\end{frame}


% % #########################################
% % #########################################


\begin{frame}{Why \textit{Should} We Care?}
    \begin{itemize}
        \item $Sm B_6$, $La_2 O_4$, $LaOFeAs$, $(CuS)$
        \item Why are we seeing states in insulating gaps of these materials? 
        \item Understanding this opens doors to finding similar high-temp SCs
    \end{itemize}
\end{frame}


% % #########################################
% % #########################################


\begin{frame}{Preexisting Models}


    \begin{itemize}
        \item Two divergent approaches:
            \begin{enumerate}
                \item Emergent Composite Particles:
                    \begin{itemize}
                        \item 'Majorana' Fermions (MFs)
                        \item Spinons
                    \end{itemize}
        
                \item 3D spin supercurrents
        \end{enumerate}
    \end{itemize}
    

\end{frame}


% % #########################################
% % #########################################

\section{The MF Model}

\begin{frame}{Bogoliubov Excitations and the Origin of MFs}
    
\end{frame}


% % #########################################
% % #########################################

\section{The Spinon Model}


\begin{frame}{Spinons}
    
\end{frame}


% % #########################################
% % #########################################

\section{Supercurrent Theory}

\begin{frame}{Supercurrents (+ Superfluidity)}
    
\end{frame}


% % #########################################
% % #########################################

\section{Takeaways}

\begin{frame}{Implications}
    
    \begin{itemize}
        \item The MF model presents a practicable rationale for states in the gap.
        
        \item Can we just keep adding (always unnecessarily) complex descriptions, or should we 'seal off' and commit to a new prescription?
        
        \item How do we reinvent condensed matter physics to both simplify and clarify what we know?
    \end{itemize}
    
\end{frame}


% % #########################################
% % #########################################


\begin{frame}{}

    \centering "A riddle, wrapped in a mystery, inside an enigma" 
    
    -- W. Churchill
    
\end{frame}


% #########################################
% #########################################


\begin{frame}[allowframebreaks]
        \frametitle{References}
        \printbibliography
\end{frame}


% #########################################
% #########################################


\end{document}