\documentclass{beamer}
\hypersetup{pdfpagemode=FullScreen}

\usetheme{cambridge} 

%%%%%%%%%%%%%%%%%%%%%%%%%%%%%%%%%%%%%%%

% bibliography stuff:
\usepackage{natbib}
\bibliographystyle{abbrv}

\setbeamercolor*{bibliography entry title}{fg=black}
\setbeamercolor*{bibliography entry location}{fg=black}
\setbeamercolor*{bibliography entry note}{fg=black}
\setbeamertemplate{bibliography item}{}
\renewcommand*{\bibfont}{\scriptsize}



% adding navigation tools:
\setbeamertemplate{navigation symbols}{} 

\addtobeamertemplate{footline}{
    \leavevmode%
    \hbox{%
    \begin{beamercolorbox}[wd=\paperwidth,ht=2.75ex,dp=.5ex,right,rightskip=1em]{mycolor}%
\usebeamercolor[fg]{navigation symbols}\insertslidenavigationsymbol%
\insertframenavigationsymbol%
\insertsubsectionnavigationsymbol%
\insertsectionnavigationsymbol%
\insertdocnavigationsymbol%
\insertbackfindforwardnavigationsymbol%
    \end{beamercolorbox}%
    }%
    \vskip0.5pt%
}{}


\definecolor{myblue}{RGB}{33,84,157}
\setbeamercolor{footline}{fg=myblue}
\setbeamerfont{footline}{series=\bfseries}



%%%%%%%%%%%%%%%%%%%%%%%%%%%%%%%%%%%%%%%

  
%%%%%%%%%%%%%%%%%%%%%%%%%%%%%%%%%%%%%%%
%%% Standard packages:
\usepackage[english]{babel}
\usepackage[utf8]{inputenc}
\usepackage{graphicx}
\usepackage{multicol}
\usepackage{subfigure}
\usepackage{listings}
\usepackage[UKenglish]{isodate}

\usepackage[scr=esstix,cal=boondox]{mathalfa} 
\usepackage{enumerate,setspace}
\usepackage{amssymb, amsthm, amsmath, amsfonts}
\usepackage{mathtools}

\usepackage{subfig}

% \usepackage{cite}
%%%%%%%%%%%%%%%%%%%%%%%%%%%%%%%%%%%%%%%


%%%%%%%%%%%%%%%%%%%%%%%%%%%%%%%%%%%%%%%
% Setup TikZ
\usepackage{tikz}
\usetikzlibrary{arrows}
\tikzstyle{block}=[draw opacity=0.7,line width=1.4cm]

\cleanlookdateon
%%%%%%%%%%%%%%%%%%%%%%%%%%%%%%%%%%%%%%%


%%%%%%%%%%%%%%%%%%%%%%%%%%%%%%%%%%%%%%%
% Author, Title, etc.
\title[Quantum Oscillations] 

}
% {%
%  {\bf Quantum Oscillations in\\ Unconventional Superconductors}%}
% }

\author[Lonzarich]
{
  \underline{Leo~Lonzarich} \and Gilbert~Lonzarich\inst{1} \and Suchitra~Sebastian\inst{1} \and Piers~Coleman\inst{2}
}

\institute[Cambridge]
{
% \inst{1}%
% Grinnell College, USA\\
\inst{1}%
Cavendish Laboratory, University of Cambridge, UK\\
\inst{2}%
Materials Theory Group, Rutgers University, USA
}

\date[CMS2021]
{Condensed Matter Theory Group S/F 2021 \hfill \today}
%%%%%%%%%%%%%%%%%%%%%%%%%%%%%%%%%%%%%%%


\begin{document}

\begin{frame}
  \titlepage
\end{frame}

\begin{frame}{Outline}
  \tableofcontents
\end{frame}


% #########################################
% #########################################
\section{Defining the Problem}


\subsection{SC Classifications}

\begin{frame}
{Superconductor Classifications}
    The well-established...
    \begin{itemize}
        \item \textsc{Gen I}:
        
            \begin{itemize}
                \item Charge neutral, e.g. lead
                \item Traditional Cooper pairing mechanism
            \end{itemize}
            
        \item \textsc{Gen II}
            \begin{itemize}
            
                \item Chemical doping, quenching long-range ferromagnetic effects
                \item Magnetic moment / local pairing of electrons
                
            \end{itemize}
    \end{itemize}
    The frontier...
    \begin{itemize}
        \item \textsc{Gen III}: % Hidden SCs!!!!!!!!!!!!!!!
            \begin{itemize}
                \item Sodium chloride with charge separation, ionic SCs
                \item Borderline ionic insulators
            \end{itemize}
        \item \textsc{Gen IV}:
            \begin{itemize}
                \item Multiferroics / magnetism and ionics
            \end{itemize}
    \end{itemize}
        
        
    % But where is the 'unconventional'?
 
\end{frame}


% #########################################
% #########################################


\subsection{States in the Gap}

% \begin{frame}
% {Thermal and Electrical Insulators}

%     \begin{itemize}
%         \item States in the Gap
%     \end{itemize}

% \end{frame}


% #########################################
% #########################################


\begin{frame}[t]{Quantum Oscillations (QOs)}
  
    \begin{itemize}
        \item Specifically w/ respect to metals:     
        \begin{itemize}
            \item Conduction electrons define a Fermi surface 
            \begin{itemize}
                \item Collection of surfaces in $k$-space separating un/occupied energy levels
                
                \item H Aoki, N Kimura \& T Terashima (2014); L Taillefer \& GGL (1988):
            \end{itemize}
            
            \begin{figure}
            \centering
            \includegraphics[scale = 0.38]{fermisurf.png}
            \label{fig:elecmuon}
            \end{figure}
    
            \item C-band electrons exhibit orbital motion on Fermi surface, quantized in $H$
            \begin{itemize}
                \item Resistivity (or capacitance) vs. $H$
            \end{itemize}
        \end{itemize}
        
    \end{itemize}

\end{frame}


% #########################################
% #########################################


\begin{frame}{The Mysterious Case of SmB$_6$}
    \begin{itemize}
        \item Insulators (with completely filled energy bands) have no Fermi surface, \textit{implying} no QOs.
        
        \item However, the Topological Kondo Insulator SmB$_6$ defies this;
            \begin{itemize}
                \item Other examples: SmS, YbB$_{12}$ La$_2$O$_4$, ruthinates (e.g., RuCl$_3$, Ca$_2$RuO$_4$), densified irons like LaOFeAs.
            \end{itemize}

        \item Why do we see this gap formation -- why coexistence of metallic and insulating properties?
        
        % Why are we seeing states in insulating gaps of these materials? 
        
        % Understanding this opens doors to finding similar high-temp SCs
        
        \item We need an approach to understanding alternative to that in literature; salient to experimental work. 
    \end{itemize}
\end{frame}


% #########################################
% #########################################


\begin{frame}{Magnetic QOs for SmB$_6$ (M Harstein et al.) }

    \begin{figure}
    \centering
    \includegraphics[scale = 0.5]{capVfreq.png}
    \label{fig:elecmuon}
    \end{figure}
    
    \centering $(B=\mu_0 H_{applied}$)

\end{frame}


% #########################################
% #########################################


\begin{frame}{SmB$_6$ QOs Cont. (BS Tan et al.)}

    \begin{figure}
    \centering
    \includegraphics[scale = 0.7]{MOA.png}
    \label{fig:elecmuon}
    \end{figure}
    
    $\alpha$-Fermi surface orbits 
    [110] Magnetic Field Direction
    24-40 T, 0.35 K

\end{frame}


% #########################################
% #########################################


\begin{frame}{Survey of Existing Theory}


    
    Three promising theories for QOs we will consider:
        \begin{enumerate}
            \item Emergent Composite Particles:
                \begin{itemize}
                    \item The 'Majorana' Fermion (MF) Model \cite{coleman1993, coleman1994}
                    \item Fermionic Composite Exciton (fCE) Model \cite{chowdhury2018}
                \end{itemize}
    
            \item Topological instability:
                \begin{itemize}
                    \item Skyrme Insulators \cite{coleman2017}
                \end{itemize}
        \end{enumerate}
    

\end{frame}


% #########################################
% #########################################
\section{The MF Model}


\begin{frame}
    
    \centering Majorana Fermions

\end{frame}


% #########################################
% #########################################


\begin{frame}{Bogoliubov Transformation $\to$ Majorana Fermions}
    
    \begin{itemize}
        \item Bogoliubov Transformation: 
        \begin{itemize}
            \item A transformation to new fermion fields; visualize the emergence of a neutral fermion via, $a \to a^+ + (aa)$; this is a Bogoliubov state.

        \end{itemize}
        
        \item Majorana fermions (MFs):
        \begin{itemize}
            \item Splitting an electron into one fluctuating charge component, and spin fluctuation degree of freedom:
            \begin{itemize}
                \item Scalar MF: $\alpha_0$, Vector MF: $\boldsymbol\alpha = (\alpha_x, \alpha_y, \alpha_z)$
            \end{itemize}
            
            \item Consequence of the tendency for particle pairs or hole pairs to form.
        \end{itemize}


        
    \end{itemize}

    \begin{figure}
    \centering
    \includegraphics[scale = 0.4]{bogstate.png}
    \label{fig:elecmuon}
    \end{figure}

\end{frame}


% #########################################
% #########################################


\subsection{Mean Field Approximations}

\begin{frame}{General Mean Field Theory}
    
    A many-body to one-body reduction:

    \begin{itemize}
        \item Consider an interaction term in the Hamiltonian that is the product $AB$ of two operator fields $A$ and $B$:
        \[
            AB = A\langle B\rangle + \langle A\rangle B - \langle A\rangle \langle B\rangle +\delta A \delta B.
        \]
        
        \item Assume $\delta A \delta B\sim 0$ (equivalent to assuming trial wavefunction).
        
        
        
        
    \end{itemize}
    
\end{frame}


% #########################################
% #########################################


% \subsection{Spin as a Majorana Fermion}

\begin{frame}{Building the Kondo Lattice Model}
    
    
    \begin{itemize}
        \item Define $c$, $f$ to be conduction and valence band fermion field operators.
        
        \item For $\mathbf{S}_i$, the spin-half local moment, the symmetric Kondo lattice Hamiltonian has the form
        \begin{center}
            $H = H_{KE} + H_{KI} = -\sum_{<ij>, \sigma} (c_{i\sigma}^+c_{j\sigma} + c_{j\sigma}^+c_{i\sigma}) + J\sum_i \mathbf{s}_i \cdot \mathbf{S}_i$
        \end{center}
        
        with $\mathbf{s} = c_{\alpha}^+ \boldsymbol\sigma_{\alpha \beta}c_{\beta}$, the conduction electron spin operator.
        
        \item The goal is conversion to real fermion fields, i.e., Majorana Fermions (MFs), achieved by applying transforms:
        \begin{itemize}
            \item $c_i \to ic_i^+$, a $\pi/2$ phase shift;
            
            \item $(c_0, \textbf{c}) \to (\alpha_0, \boldsymbol\alpha)$; $\alpha_0$ is the scalar MF field and $\boldsymbol\alpha = (\alpha_x, \alpha_y, \alpha_z)$ is the vector MF field.
        \end{itemize}
        
        
    \end{itemize}
    
    
    
    
    
\end{frame}


% #########################################
% #########################################


\begin{frame}{Leading to Spin Decoupling}
    
    \begin{itemize}
        \item KE in Hamiltonian then describes four degenerate Majorana bands
        \[
        {\displaystyle
        H_{KE} \to -it\sum_{<ij>} (\alpha_{i0}\alpha_{j0} + \boldsymbol\alpha_i \cdot \boldsymbol\alpha_j),
        }
        \]
        % and we have our MF fields:
        % \[
        % {\displaystyle
        %     \alpha_{i0} = \frac{1}{2i}(-c_{i\downarrow}^+ + c_{i\downarrow}), \ \ 
        %     \alpha_{ix} = \frac{1}{2}(c_{i\uparrow}^+ + c_{i\uparrow}),
        % }
        % \]
        and the KI term becomes, for $\boldsymbol\eta = \boldsymbol\sigma /\sqrt{2}$,
        \[
        {\displaystyle
        \frac{J}{2} \sum_i \left(\alpha_{i0} \boldsymbol\alpha_i \cdot (\boldsymbol\eta_i\times\boldsymbol\eta_i) - \frac{1}{2}(\boldsymbol\alpha_i \cdot \boldsymbol\eta_i)^2\right) + constant.
        }
        \]
        With a complete transformation of the Kondo $H$, two mean field decouplings naturally arise
        
    \end{itemize}
    
\end{frame}


% #########################################
% #########################################


\subsection{The Majorana Fermion MFT}

\begin{frame}{The Majorana Fermion MFT}
    
    
    \begin{itemize}
        \item In the nonmagnetic limit, and neglecting numerical factors, the Majorana MFT falls out:

        \begin{center}
            $\boldsymbol s \cdot \boldsymbol S \to 2 \langle \boldsymbol\alpha_i \cdot \boldsymbol\eta_i\rangle \boldsymbol\alpha_i \cdot \boldsymbol\eta_i - \langle \boldsymbol\alpha_i \cdot \boldsymbol\eta_i\rangle^2$
        \end{center}
        
        \item Thus, scalar Majorana field is ungapped. 
        \item Recasting $\langle \boldsymbol\alpha_i \cdot \boldsymbol\eta_i\rangle$ in terms of original fermion fields, we find triplet pairing correlations --- crucial to suppressing hybridization:
        % and therefore putting states in the gap
        \begin{center}
            $\langle \boldsymbol\alpha_i \cdot \boldsymbol\eta_i\rangle = 2\langle c_{i\uparrow}^+f_{i\uparrow}+ c_{i\uparrow}f_{i\uparrow}^+ \rangle + \left[ 
            \langle c_{i\downarrow}^+f_{i\downarrow}+ c_{i\downarrow}f_{i\downarrow}^+ \rangle +
            \langle {\color{orange} c_{i\downarrow}^+f_{i\downarrow}^+} + {\color{orange}c_{i\downarrow}f_{i\downarrow}} \rangle
            \right]$
        \end{center}
        
    \end{itemize}
    
\end{frame}


% #########################################
% #########################################


\begin{frame}{The Majorana Fermion MFT Cont.}
    
    \begin{itemize}
        \item Transformation shows both hybridization and triplet pairing correlations in KI term of $H$:
        \begin{itemize}
            \item Superconducting state with low phase
            stiffness;
            
            \item SC suppressed by fluctuations, defects, and/or emergent 3D order parameter;
            
            \item Fluctuations due to nearly-immobile f-electrons, or scarcity of holes. 
        
        \end{itemize}
        
        \item Novelty in using a dual-band model requires that we focus only on spin triplet pairing;
    \end{itemize}
    
\end{frame}


% #########################################
% #########################################


\begin{frame}{Beyond Mean Field Theory}
    
    \begin{itemize}
        \item We can simply try a different MFT; looking for a better trial wave function:
        \begin{itemize}
            \item Weiss, Hybridization, Hybrid + triplet pairing, RKKY
        \end{itemize}
        %Of course, we try to do this, but we may run out of ideas that usefully represent the wave function. 
        
        \item Is there a systematic way of improving upon a given MFT? Yes, various:
        \begin{enumerate}
            
            \item Systems Analysis; linear response, feedback, Langevin noise.

            
            \item Take a textbook approach and look for leading order perturbation theory corrections.
            %At minimum, this can be made self-consistent
            
            \item Effective Field Theory (EFT): treat order parameter as a starting point for a new quantum field, develop an effective Lagrangian description. 
            \begin{itemize}
                \item Products of fermion fields.
                % instead of a single fermion field
            \end{itemize}

        \end{enumerate}

    \end{itemize}
    
\end{frame}


% #########################################
% #########################################
\section{The fCE Model}


\begin{frame}
    
    \centering Fermionic Composite Excitons

\end{frame}


% #########################################
% #########################################

% \begin{frame}{Hybridization of c-electron and f-electron Bands}
    
%     \begin{figure}
%     \centering
%     \includegraphics[scale = 0.5]{gap.png}
%     \label{fig:elecmuon}
%     \end{figure}
    
% \end{frame}


% % #########################################
% % #########################################


% \begin{frame}{f-electron Hopping and Hybridization with c-states}
    
%     \begin{figure}
%     \centering
%     \includegraphics[scale = 0.5]{hybrid.png}
%     \label{fig:elecmuon}
%     \end{figure}
    
% \end{frame}


% #########################################
% #########################################


\begin{frame}{Gapless Fermionic Excitations (single-band)}

    \begin{itemize}
        \item Via Gutzwiller projection, enforce $n_f=1$ by multiplying the creation operator by a bosonic condition that an opposite spin state be absent: 
        \begin{center}
            $f_{\sigma}^+ \to g_{\sigma}^+ = (1-n_{-\sigma})f_{\sigma}^+$
        \end{center}
        for a slave boson $b = 1-n_{-\sigma}$. 
        
        \item We equivalently have
        \begin{center}
            $g_{\sigma}^+ = f_{\sigma}^+ - f_{-\sigma}^+f_{-\sigma}f_{\sigma}^+$,
        \end{center}
        a charge-neutral spinon.
    \end{itemize}
    
\end{frame}


% #########################################
% #########################################


\begin{frame}{Gapless Fermionic Excitations (Two-band)}
    
    \begin{itemize}
        \item Returning to the Two-band Kondo problem:
            \begin{itemize}
                \item In the absence of the interactions, a spin-zero excitonic excitation is described by $c_{\sigma}^+f_{\sigma}$ -- a charge-zero boson.
                
                \item To include interactions, we consider the following fermionic composite 'exciton' (fCE):
                \[
                    h_{\sigma}^+ = (1-n_{-\sigma})c_{\sigma}^+ = c_{\sigma}^+ - f_{-\sigma}^+f_{-\sigma}c_{\sigma}^+.
                \]
            
            \end{itemize}
        
    \end{itemize}
    
    
    
    
    \begin{figure}[htbp!]%
        \centering
        \subfloat{{\includegraphics[width=4cm]{holon.png} }}%
        \qquad
        \subfloat{{\includegraphics[width=4cm]{fce.png} }}%
        \label{fig:example}%
    \end{figure}


\end{frame}


% #########################################
% #########################################


\begin{frame}{The Potential of fCEs}
    
    \begin{itemize}
        \item As with conventional excitons, the $h_{\sigma}^+$ are states in the gap, and because they are fermions, they can precipitate QOs;
        \begin{itemize}
            \item f-band electron-hole pairing make $h_{\sigma}^+$ a three-fermion field with energy in insulating gap.
            
            \item Forms the necessary Fermi surface.

        \end{itemize}
        
    \item Scalar MF is a superposition of a particle and a hole giving charge zero; fEC is product of particle-hole fields. 
    
    \item Suppressed double occupancy by locking f-band electrons and holes together (holon). 
        
    \end{itemize}
\end{frame}


% #########################################
% #########################################
\section{Supercurrent Theory}

\begin{frame}
    
    \centering Skyrme Insulators and \\
    \centering Topological Instability
\end{frame}


% #########################################
% #########################################


\begin{frame}{Skyrme Insulators: Understanding the Supercurrent}
    
    \begin{itemize}
        \item Cyclotron Motion $ \not \Leftrightarrow $ Insulating Behaviour;
        \begin{itemize}
            \item So the bulk must somehow break gauge invariance $\to$ dissolve supercurrents $\to$ fail SC.
            
        \end{itemize}
        
        \item Skyrmions:
        \begin{itemize}
            \item Form an unpinned liquid akin to vortex liquids of type II SCs.
            
            \item Density of Skyrmions is $n_s = B/ \phi_0$, so thermal conductivity is $\kappa \propto H$.
            % for skyrmion liquid
        \end{itemize}
        
        \item Skyrme insulator: A stacking of two-dimensional Skyrmion configurations, producing a dielectric.
        
        \item Dielectric facilitates breaking of gauge invariance;
        
        \begin{itemize}
            \item Neutral quasiparticles, dissolving current operator, but...
            \item Enabling diamagnetism.
        \end{itemize}
    \end{itemize}
    
    % supercurrent is a non-dissipative, superconducting current
    
    
\end{frame}


% #########################################
% #########################################


\begin{frame}{Topological (In)Stability: A String Analogy}
    
    
    \begin{itemize}
        \item Gauge invariant current $J= q\frac{n}{m}(\hbar \nabla \theta -qA)$, with order parameter $\Psi(r) = |\Psi| \exp(i\theta)$ from Bogoliubov theory;
        \begin{itemize}
            \item Supercurrent density $n=|\Psi|^2$.
        \end{itemize}
        \item The order parameter of a conventional SC lies on a circular manifold in $\mathbb{C}$;
        \begin{itemize}
            \item Topologically stable winding number; a string about a rod.
        \end{itemize}
        
        \item For TKI SmB$_6$, the order parameter, $M(r)$, lies on a spherical manifold in $\mathbb{C}\times \mathbb{R}$;
        \begin{itemize}
            \item Unprotected winding; a string about a sphere.
        \end{itemize}
    \end{itemize}
    
    
\end{frame}


% #########################################
% #########################################


\begin{frame}{A String Analogy Cont.}
    
    \begin{figure}
    \centering
    \includegraphics[scale = 0.45]{top1.png}
    \label{fig:top}
    \end{figure}
    
    
    
\end{frame}


% #########################################
% #########################################


\begin{frame}{Visualizing 4D Supercurrents}
    
    \begin{figure}[t!]
    \centering
    \includegraphics[scale = 0.43
    ]{top2.png}
    \label{fig:top}
    \end{figure}
    

\end{frame}


% #########################################
% #########################################


\begin{frame}{Topological Instability}
    
    \begin{itemize}
        \item $\phi \to 0$ such that $J\to 0$, giving the expected current instability.
        \begin{itemize}
            \item $M$ tends towards $\Delta x = 0$ on average because it sees reductions in energy for doing so.
            % (a state exhibiting current is not in equilibrium).
        \end{itemize}
        
        \item Thus, consider the missing degree of freedom for $|\Psi|$ as being an infinite resistance to equilibration.
        
        \item Failed SC $\Leftrightarrow$ transformation into a Skyrme insulator (dielectric)
        
        \item Recalling, $\kappa \sim H$, we have implication of a flux liquid:
        \begin{itemize}
            \item Skyrmion insulators verifiable if we find anomalous conductivity is perpendicular to $H$ \cite{coleman2017}.
        \end{itemize}
    \end{itemize}

\end{frame}


% #########################################
% #########################################
\section{Takeaways}


\begin{frame}{Implications}
    
    \begin{itemize}
        \item Both gapless scalar MF-liquids and fCEs present practicable explanations for states in the gap.
        
        \item Reduced mobility do to $c$ and $f$ electron pairing -- $f$ lowers mobility; 
        \begin{itemize}
            \item Reduced phase stiffness as in RVB theory may suppress long-range SC order.
        \end{itemize}
        
        \item As it concerns experiment:
        \begin{itemize}
            \item Many long/short-range order states appear possible; hydrostatic pressure tuning of very pure samples (SmB$_6$ and SmS) may recover these.
            % Perhaps pressure brings specimen is close to a Majorana state and we simply need more pressure to reach it 
            
            \item States in the insulating gap imply large, linear heat capacities; thermal -- not electrical -- conductivity: We can measure this!
        \end{itemize}
        
        \item This calls for renewed study of other well-known TKIs like SmS and FeSi.

    \end{itemize}
    
\end{frame}


% #########################################
% #########################################


\begin{frame}{}

    \centering "A riddle, wrapped in a mystery, inside an enigma" 
    
    -- W. Churchill
    
\end{frame}


% #########################################
% #########################################


\begin{frame}[allowframebreaks]
        \frametitle{References}
        
        \bibliography{uoc}

\end{frame}


% #########################################
% #########################################


\end{document}